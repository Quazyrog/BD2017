\documentclass[a4paper, 10pt]{article}
\nonstopmode
\usepackage[margin=1cm, top=2cm, bottom=2cm]{geometry}
\usepackage{fontspec}
\usepackage[polish]{babel}
\usepackage{graphicx}
\usepackage{tikz}
\usepackage{xcolor}
\usepackage{listings}
\usetikzlibrary{er,positioning,arrows.meta}

\author{Wojeciech Matusiak \textless wm382710@students.mimuw.edu.pl\textgreater}
\title{Zadanie zaliczeniowe z baz danych}

\lstset{
    basicstyle=\normalsize\ttfamily,
    breakatwhitespace=false,
    breaklines=true,
    commentstyle=\color[rgb]{0, 0.4, 0},
    extendedchars=true,
    frame=single,
    keepspaces=true,
    keywordstyle=\color{blue},
    numbers=left,
    numbersep=5pt,
    numberstyle=\footnotesize,
    rulecolor=\color{black},
    showstringspaces=false,
    stringstyle=\color{red},
    literate={ę}{{\k e}}1 {ó}{{\'o}}1 {ą}{{\k a}}1 {ś}{{\'s}}1 {ł}{{\l}}1 {ż}{{\.z}}1 {ź}{{\'z}}1 {ć}{{\'c}}1 {ń}{{\'n}}1,
    xleftmargin=5mm
}

\lstdefinelanguage{ODL}{
  morekeywords={
    interface,
    attribute,
    reference,
    method
  },
  sensitive=true % keywords are not case-sensitive
}


\begin{document}
    \maketitle

    \section{Projekt bazy danych}

    \subsection{Diagram ER}
    \begin{figure}[h]
        \centering
        \begin{tikzpicture}[auto,node distance=1.5cm]
            %% LogEntries entity
            \node[entity] (LogEntries) {LogEntries}
                [grow=left, level distance=3cm, sibling distance=9mm]
                child[grow=up, level distance=2cm, xshift=40mm] {node[attribute] {\underline{id}}}
                child[grow=up, level distance=2cm, xshift=28mm] {node[attribute] {size}}
                child[grow=up, level distance=2cm, xshift=13mm] {node[attribute] {time}}
                child[grow=up, level distance=2cm, xshift=-10mm, yshift=-3mm] {node[attribute] {timeToServe}}
                child[yshift=15mm, xshift=4mm, yshift=-2mm] {node[attribute] {request}}
                child[yshift=12mm, xshift=-2mm] {node[attribute] {remoteAddress}}
                child[yshift=12mm] {node[attribute] {urlPath}}
                child[yshift=12mm, xshift=4mm] {node[attribute] {status}};
                
            %% uploadedFrom relationship
            \node[relationship] (uploadedFrom) [right = of LogEntries] {uploadedFrom};
            
            %% LogFiles entity
            \node[entity] (LogFiles) [below right = of uploadedFrom] {LogFiles}
                [grow=right, level distance=4cm, sibling distance=9mm]
                child {node[attribute] {\underline{id}}}
                child {node[attribute] {uploadDate}}
                child {node[attribute] {uploadFormat}}
                child {node[attribute] {duplicatesSkipped}}
                child[xshift=-10mm] {node[attribute] {comment}};
                
            %% fromServer relationship
            \node[relationship] (fromServer) [below left = of LogFiles] {uploadedFrom};
                
            %% Server entity
            \node[entity] (Servers) [left = of fromServer] {Servers}
                [grow=down, level distance=1cm, sibling distance=3cm]
                child[yshift=15mm] {node[attribute] {\underline{name}}}
                child[xshift=-25mm, yshift=5mm] {node[attribute] {description}}
                child[xshift=-23mm] {node[attribute] {defaultLogFormat}};
              
            %% uploadedFrom relationship (ARROWS)
            \draw (uploadedFrom) 
                edge node {} (LogEntries)
                edge[-{Latex[length=2mm,width=3mm]}] node {} (LogFiles);
                
            %% uploadedFrom relationship (ARROWS)
            \draw (fromServer) 
                edge node {} (LogFiles)
                edge[-{Latex[length=2mm,width=3mm]}] node {} (Servers);
                
            
            %% Filters entity
            \node[entity] (Filters) [below = 2cm of Servers] {Filters}
                [grow=left, level distance=3cm, sibling distance=1cm]
                child {node[attribute] {\underline{name}}}
                child {node[attribute] {queryString}};
                
            %% ValuesLists entity
            \node[entity] (ValuesLists) [right = 5cm of Filters] {ValuesLists}
                [grow=left, level distance=3cm, sibling distance=1cm]
                child {node[attribute] {\underline{id}}}
                child {node[attribute] {name}};
                
            %% fromList relationship
            \node[double, relationship] (fromList) [right = of ValuesLists] {fromList};
                
            %% ValuesListEntries entity
            \node[double, entity] (ValuesListEntries) [below = of fromList] {ValuesListEntries}
                [grow=left, level distance=3cm, sibling distance=1cm]
                child {node[attribute] {\underline{value}}};
                
            %% fromList relationship (ARROWS)
            \draw (fromList) 
                edge[-{Latex[length=2mm,width=3mm]}] node {} (ValuesLists)
                edge node {} (ValuesListEntries);
        \end{tikzpicture}
        \label{fig:erd}
        \caption{Diagram związków encji dla aplikacji}
    \end{figure}
    
    
    \subsection{Skrótowy opis encji występujących w bazie}
    
    \subsubsection{LogEntries}
    Obiekt \texttt{LogEntry} odpowiada pojedynczemu wpisowi w pliku logu. Każdy posiada unikatowy identyfikator na wewnętrzne potrzeby. Kolejne wpisy w obrębie wgranego pliku powinny mieć rosnące identyfikatory. Wszystkie poza atrybuty  \texttt{id} mogą nie mieć wartości, kiedy we wgrywanym pliku nie pojawiały się odpowiednie informacje
    
    \begin{description}
        \item[id] Unikatowy identyfikator danego wpisu
        \item[size] rozmiar odpowiedzi z nagłówkami, jeżeli dostępny, a jak nie, to bez nagłówków, chyba, że ten też jest niedostępny [\texttt{\%b} lub \texttt{\%B}]
        \item[time] czas, z którego pochodzi dany wpis [\texttt{\%t}]
        \item[timeToServe] czas przetwarzania żądania [\texttt{\%{FORMAT}T} lub \texttt{\%D}]
        \item[request] typ żądania (GET, POST itp.) [\texttt{\%m} lub zgadywane z \texttt{\%r}]
        \item[remoteAddress] adres, z którego pochodzi rządanie [\texttt{\%h}]
        \item[urlPath] fragment URL zawierający ścieżkę do żądanego zasobu (bez ,,query string'') [\texttt{\%U} lub zgadywane z \texttt{\%r}]
        \item[status] status odpowiedzi [\texttt{\%s} lub \texttt{\%>s}]
    \end{description}
    
    \subsubsection{LogFile}
    Obiekt \texttt{LogFile} akcji wgrania pliku logu do aplikacji. 
    
    \begin{description}
        \item[id] identyfikator operacji wgrania logu
        \item[uploadDate] data wgrania pliku
        \item[uploadFormat] format używany podczas parsowania wpisów z pliku (np. \texttt{\%h \%l \%u \%t "\%r" \%>s \%b})
        \item[duplicatesSkipped] liczba wpisów, które zostały rozpoznane jako duplikaty wpisów znajdujących się już w bazie i przez to zostały pominięte
        \item[comment] opcjonalny komentarz użytkownika do wgrywanych danych
    \end{description}
    
    \subsubsection{Servers}
    Według (obecnych) założeń, aplikacja ma umożliwiać pracę z różnymi serwerami i zapewniać pewną separację (np. kiedy wyszukujemy konkretne wpisy, to to wyszukiwanie jest zawsze ograniczone do konkretnego serwera). Obiekt \texttt{Server} stanowi właśnie reprezentacje takiego serwera.
    
    \begin{description}
        \item[name] unikatowa nazwa serwera nadana przez użytkownika w momencie dodawania serwera do aplikacji
        \item[description] opis serwera
        \item[defaultLogFormat] domyślna wartość \texttt{uploadFormat} dla kolejnych operacji wgrywania logów z tego serwera
    \end{description}
    
    \subsection{Filters, ValuesLists i ValuesListEntries}
    Zapisane filtry użytkownika. Obiekt \texttt{Filter} reprezentuje pojedynczy taki filtr i jest przechowywany w postaci napisu zapytania, który użytkownik wprowadza w polu wyszukiwania. (Format zostanie ustalony później.)
    
    Klasa \texttt{ValuesList} umożliwia przechowywanie predefiniowanych list wartości, do wykorzystania w wyszukiwaniu (na przykład lista kilku wrednych adresów z Republiki Chińskiej). Encje ValuesLists oraz ValuesListEntries służą właśnie do przechowywania tych list w bazie danych.
    
    \subsection{Opis w ODL}
    \lstinputlisting[language=ODL]{ODL.txt}
\end{document}
